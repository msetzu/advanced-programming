\documentclass[10pt, a4paper]{article}

\usepackage[eng,exjobb]{KTHEEtitlepage}
\usepackage[cp1252]{inputenc}
\usepackage{sectsty}
\usepackage{titlesec}
\usepackage[margin=.8in]{geometry}
\usepackage[obeyspaces,spaces]{url}
\usepackage{svg}
\usepackage{listings}
\usepackage{minted}
\usepackage{charter}

\definecolor{bl}{RGB}{19,65,107}
\definecolor{rd}{RGB}{163,54,67}
\newcommand{\file}[1]{{\color{bl}\underline{{\footnotesize \path{#1}}}}}
\newcommand{\code}[1]{{\color{bl}\texttt{#1}}}

\titleformat{\section} {\bfseries\scshape\color{bl}}{}{}{}
\titleformat{\subsection} {\bfseries\scshape\color{bl}}{}{}{}

\begin{document}
    \ititle{Advanced Programming}
    \isubtitle{Final term paper submission}
    \idate{June 2017}
    %\irefnr{TRITA-EE 2006:666}
    \iauthor{Mattia Setzu, 544666}
    \makeititle

    \section*{Implementation}
    The following implementation models the problem as following: \emph{i)} nodes are modelled by \emph{Node}, \emph{ComputationalNode} and \emph{InputNode}, the last one implementing pure values (i.e. pure matrices) too; \emph{ii)} a node is a mapping from a name to a value (input) or an operation (computational), enriched by an input ordered set; \emph{iii)} operations are disjoint from nodes, hence should be modelled in separate classes (\emph{Operation}); \emph{iv)} the code and graph generators (\emph{PyFactory}, \emph{GraphFactory}) receive well-formed graphs.
    We add the web references at the end of this introductory page
    \footnote{https://stackoverflow.com/questions/27677256/java-8-streams-to-find-the-duplicate-elements,
		 		http://json.org
    			https://stackoverflow.com/questions/12643009/regular-expression-for-floating-point-numbers
    			http://www.adam-bien.com/roller/abien/entry/java\_8\_streaming\_a\_string
    			https://stackoverflow.com/questions/34879086/java-8-stream-api-filter-on-instance-and-cast}.

    \section{Exercise 1}
    Design a hierarchy of classes to represent nodes and graph objects.
    The classes should provide suitable constructors for building nodes and graphs.
    

\file{file: ./final/src/main/java/com.github.msetzu.pa.graph.Node.java}
\inputminted[tabsize=1,fontsize=\small]{java}{./final/src/main/java/com/github/msetzu/pa/graph/Node.java}
    
\file{file: ./final/src/main/java/com.github.msetzu.pa.graph.ComputationalNode.java}
\inputminted[tabsize=1,fontsize=\small]{java}{./final/src/main/java/com/github/msetzu/pa/graph/ComputationalNode.java}

\file{file: ./final/src/main/java/com.github.msetzu.pa.graph.InputNode.java}
\inputminted[tabsize=1,fontsize=\small]{java}{./final/src/main/java/com/github/msetzu/pa/graph/InputNode.java}

\file{file: ./final/src/main/java/com.github.msetzu.pa.graph.Matrix.java}
\inputminted[tabsize=1,fontsize=\small]{java}{./final/src/main/java/com/github/msetzu/pa/graph/Matrix.java}

\file{file: ./final/src/main/java/com.github.msetzu.pa.graph.Operation.java}
\inputminted[tabsize=1,fontsize=\small]{java}{./final/src/main/java/com/github/msetzu/pa/graph/Operation.java}
    
    
    \section{Exercise 2}
    Implement a recursive descent parser, which takes a JSON representation of a graph and builds the corresponding graph. For example, the above computational graph is represented as \textbf{[..]}.
       
    \begin{minted}[tabsize=2, fontsize=\small]{javascript}
    { "a": {"type": "input", "shape": [1,1]},
    "b": {"type": "input", "shape": [1,1]},
    "c": {"type": "comp", "op": "sum", "in": ["a", "b"]},
    "d": {"type": "comp", "op": "sum", "in": ["b", [[1]]]},
    "e": {"type": "comp", "op": "mult", "in": ["c", "d"]}}
    \end{minted}
    
    Notice that not all the operations are possible, for example summing two vectors of different shapes.
    The Graph class should provide the methods: isDAG, and isValid to check whether all the variables are defined and if all the operations are computable.
    For checking the compatibility, there should be a table specifying the signature for each function, for example:
    \begin{minted}[tabsize=2, fontsize=\small]{ocaml}
    sum: ([n, m], [n, m]) -> [n, m]
    mult: ([n, m], [m, k]) -> [n, k]
    \end{minted}

    \paragraph{De-coupling} In order to de-couple the graph from the operations, we provide the validity constraints on the enumeration holding the operation themselves: we trivially check for dimension correspondence.
    An intermediate representation of the graph is stored in the map nodes in the form \code{name -> {TYPECASE -> "INPUT|COMP", SHAPE -> "m,n"}, ...}.
    Such representation is then accessed by \code{GraphFactory} to generate the graph.
    \vspace{2mm}
    
\file{file: ./final/src/main/java/com.github.msetzu.pa.graph.json.Token.java}
\inputminted[tabsize=1,fontsize=\small]{java}{./final/src/main/java/com/github/msetzu/pa/graph/json/Token.java}
    
    
\file{file: ./final/src/main/java/com.github.msetzu.pa.graph.Operation.java}
\inputminted[tabsize=1,fontsize=\small]{java}{./final/src/main/java/com/github/msetzu/pa/graph/Operation.java}

\file{file: ./final/src/main/java/com.github.msetzu.pa.graph.json.JSONParser.java}
\inputminted[tabsize=1,fontsize=\small]{java}{./final/src/main/java/com/github/msetzu/pa/graph/json/JSONParser.java}

\file{file: ./final/src/main/java/com.github.msetzu.pa.graph.json.factories.GraphFactory.java}
\inputminted[tabsize=1,fontsize=\small]{java}{./final/src/main/java/com/github/msetzu/pa/graph/json/factories/GraphFactory.java}
    
    \section{Exercise 3}
    Use polymorphism to implement a compiler that transforms a graph into executable code in a programming language.
    Use a template for each function, that represents the code to be generated for the body of the function.
    The resulting code should correspond to a function, which takes as arguments variables corresponding to the inputs of the graph and returns an array with all the outputs of the graph.
    Provide the code generated for the example in the Introduction.

    \paragraph{Polymorphism} We built the program on a top-down approach: starting from the top we add the statements necessary to build the current node based on its operation: in order to do so we delegate to our \code{OperationCompiler} implementation to whom we provide the node.
    We abstract over the operations by providing a general \code{OperationCompiler}  that acts as an hub for the various compilers.  

\file{file: ./final/src/main/java/com.github.msetzu.pa.graph.factories.PyFactory.java}
\inputminted[tabsize=1,fontsize=\small]{java}{./final/src/main/java/com/github/msetzu/pa/graph/factories/PyFactory.java}

\file{file: ./final/src/main/java/com.github.msetzu.pa.graph.compilers.OperationCompiler.java}
\inputminted[tabsize=1,fontsize=\small]{java}{./final/src/main/java/com/github/msetzu/pa/graph/compilers/OperationCompiler.java}

\file{file: ./final/src/main/java/com.github.msetzu.pa.graph.compilers.SumCompiler.java}
\inputminted[tabsize=1,fontsize=\small]{java}{./final/src/main/java/com/github/msetzu/pa/graph/compilers/SumCompiler.java}

\file{file: ./final/src/main/java/com.github.msetzu.pa.graph.compilers.MulCompiler.java}
\inputminted[tabsize=1,fontsize=\small]{java}{./final/src/main/java/com/github/msetzu/pa/graph/compilers/MulCompiler.java}

\subsection{Generated code}
\begin{minted}[tabsize=1,fontsize=\small]{python}
import numpy

if __name__ == "__main__":
	a = numpy.matrix(sys.argv[1])
	b = numpy.matrix(sys.argv[2])
	d = b + numpy.matrix('[[1]]')
	c = a + b
	e = numpy.matmul(c, d)
	return [e]
\end{minted}
    
    
    \section{Exercise 4}
    Extend the code generator in order to perform code optimizations.
    In particular, the generator should identify cases where multiple operations can be fused into one, for example: should be compiled into code that performs the addition a + b + c with a single nested loop.
    \emph{Hint: use a template to represent the transformation.}
    
    \paragraph{Algorithm} We implement a top-down optimization algorithm similar to the processing phase of the Dijkstra min-distance: starting from the top, we split the lower nodes in two sections: boundary and not boundary, that is the immediate children and not immediate children.
    Then we merge the boundary nodes through \code{merge(node, graph)} whenever possible and advance the boundary recursively until we reach the bottom (the input nodes).
    
\file{file: ./final/src/main/java/com/github/msetzu/pa/graph/optimizers/Optimizer.java}
\inputminted[tabsize=1,fontsize=\small]{java}{./final/src/main/java/com/github/msetzu/pa/graph/optimizers/Optimizer.java}
    
    \section{Exercise 5}
        
    \paragraph{C++ templates}\emph{C++} allows the definition of dependent types, i.e. types defined w.r.t. parametric values.
    This technique is aimed at generating suitable code (at least) partially evaluated at compile time.
    By inserting statically-known values to the template we leverage the compile-time speed and provide a concrete model on which the compiler can optimize through static analysis.
    It is then trivial to show how this technique comes in handy the closer the input values are to the template ones: the lesser the distance, the lesser the run-time computation.
    On the other hand, to find the model closest to the run time inputs is often a non-trivial task, and lowers the generality of the overall program.
    
    \paragraph{LINQ expression trees} LINQ is a \emph{.NET} component exploiting expression trees to abstract over structured data modelled in a non-Object Oriented fashion, providing easier integration.
    Common data models are xml files, databases and objects.
    Possible expressions include \code{select}, \code{where}, \code{sum}, \code{avg}, \code{join} and \code{foreach} to allow iteration.
    Set operators are provided, such as \code{intersect}, \code{union}, etc.
    Operators are composed and built through lambda expressions implemented with the parametric type \code{Expression<T>}} and stored by the platform in efficient in-memory data structures, the expression trees.
	Then it is trivial to compute over the above data models through the abstraction layer provided by \emph{.NET}.
    Moreover, \emph{LINQ} is enriched by a type-inference tool, hence is able to infer and assign types to expression to/from the supported data models.
    By typing structured untyped data we are able to guarantee some constraints on type safety on the input data.
    We may also declare new structures and create anonymous ones to use in throw-away computations: such anonymous types are also automatically enriched with common methods (get, set, etc.)
\end{document}